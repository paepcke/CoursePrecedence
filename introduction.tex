\section{Introduction}
{\color{red}[LOCKED BY ANDREAS]}


%% - What is the problem?
%% - Why is it important?
%% - Why is it hard?
%% - How are current solutions insufficient?
%% - What we do

The US higher education system is unique in the world in the extent to
which it expects undergraduates to explore a variety of courses before
committing to a field of study. In contrast with virtually all other
national postsecondary systems, in which students enter schools and
programs with relatively structured curriculums, US undergraduates are
encouraged to explore a variety of academic options through an
iterative course search and selection process. We call a student's
eventual sequence of course choices a {\it pathway}.

While enriching when successful, this exploration is fateful. It can
stall timely progress toward graduation \cite{bailey2015redesigning}
and thereby be financially burdensome. The absence of explicitly
designed pathways can create path dependencies that neither students
nor faculty fully understand \cite{chambliss2014college}. An
inopportune set of choices can unexpectedly preclude settling on some
of the majors that were available earlier. 

Furthermore, demographic differences between students interacting with
pathway choices can reinforce inequality
\cite{armstrong2013paying}. Ethnographic research suggests that
students tend to select courses based on partial, poorly integrated
information, in light of logistical constraints, and personal
preferences that have little directly to do with academics
\cite{nathan2006my,rosenbaum2011complexities, rosenbaum2007after}.

Yet, understanding the totality of curriculum options is challenging
not only for students. Other stake holders in the enterprise of higher
education also need convenient insight into the
curriculum. Instructors, department chairs, deans, provosts, and even
Boards of Trustees can usually not easily gain curriculum related
insights. One challenge in mitigating this lack is that each of these
potential beneficiaries of improvement require investigative equipment
of different focal lengths.

Day to day, {\bf academic advisors} on the ground lack tools for
understanding the characteristics of the numerous majors and programs
they are tasked to explain. For example, it is not trivial to identify
majors whose requirements have students take courses primarily within
one department. Requirement structures for majors such as Computer
Science tend to impose this more intradepartmental regime. Other
majors, such as History, instead have students consume courses such as
economics and political science. Insight into such differences are
needed for helping advisers suggest pathways that match their student
clients' interests and temperaments.

{\bf Instructors}, in contrast, require information local to their
specific course offerings. Yet they cannot easily answer the question
{\it Which majors, and preceding courses feed students into my class?}

{\bf Department chairs} and {\bf deans} are for different reasons in
need of knowing the migration paths of students in and out of
departments. Questions such as {\it Does our school offer a rich set
  of non-major service courses to the rest of the university?}, and
{\it are students from across campus uniformly taking advantage of
  such courses?} are difficult to answer.

Beyond such specific needs, the myriad of possible pathways through
curricula also impedes efforts by university administrators to
encourage behaviors that lead to outcomes deemed desirable by
their institutions. Such goals might emphasize persistence,
intellectual breadth, or readiness for the workplace.

The information for these needs is present in all teaching
institutions. But it is embodied in the tables of databases to which
few have access, and which would be too difficult to navigate even if
access was available for all the stake holders. Many universities have
specialized staff that generate specifically requested reports for
individuals in high level academic positions\footnote{We acknowledge
  here our own version of such an {\it Institutional Research and
    Decision Support} unit, which has provided us with numerous
  insights into both, information needs, and the semantics of data at
  our institution.} But they cannot interact with all the individuals
in need of insight at varying levels of detail.

Some universities have created software that surfaces a selection of
the needed information broadly to its institution's clients
(e.g. \cite{carta}). But these tools focus primarily on the narrower
needs of individual students, and therefore lack the ability to
provide both overview and detailed insight.

We report here on {\it Via}, which makes inroads into some of these
vexing problems. Our approach is to provide a zoomable, investigative
instrument for large scale, qualitative curriculum exploration. The
system builds on graph theory, and provides a visual interface to
viewing curricula. Graph approaches have been applied to a number of
domains and tasks, such as detecting communities
\cite{PhysRevE.70.056104}, collecting materials for survey articles
\cite{ji2015}, the augmentation of collaborative recommendation
records \cite{huang2005}, predicting future collaborations between
scholars \cite{liben2007}, and suggesting drug interactions
\cite{zitnik2018}.

We rely primarily on student enrollment records. Our methods are
therefore applicable to any institution that maintains at least this
level of record keeping. We explain how the availability of additional
data, such as demographic information greatly adds to the results that
can be achieved.

To first approximation, we convert enrollment records to directed
graphs, with nodes modeling courses, links modeling sequential
enrollment, and link weights modeling enrollment volume. We partition
the graphs by courses' department membership. An existing graphing
tool \cite{shannon2003cytoscape} exposes more, or less information,
depending on chosen zoom levels.

While the concept is easy to explain, details arise that are not
immediately obvious. In addition, the most informative mapping from
enrollments to graphs depends on the questions the graph is intended
to answer. Resulting mappings can vary widely.

After related work, and a brief introduction to our dataset,
Section~\ref{sec:visualization} introduces {\it Via}'s visual
component. In Section~\ref{sec:mapping} we explain alternative graph
mapping options, and their implications for questions that can be
investigated. Sections~\ref{sec:stud_matrix}/\ref{sec:aggr_matrix}
detail the necessary processing of enrollment data in preparation to
their ingestion into the visualization component.


