\section{Introduction}
%%{\color{red}[LOCKED BY ANDREAS]}


%% - What is the problem?
%% - Why is it important?
%% - Why is it hard?
%% - How are current solutions insufficient?
%% - What we do

US higher education is unique in the world in the extent to
which schools expect undergraduates to explore a variety of courses before committing to a field of study. In contrast with virtually all other national postsecondary systems, in which students enter schools and programs with relatively structured curricula, US undergraduates are encouraged to explore a variety of academic options through an iterative course search and selection process. We call a student's
eventual sequence of course elections a {\it pathway}. Pathways are notoriously poorly instrumented for observation by students, educators, administrators or researchers \cite{chambliss2014college}.

While enriching when successful, the contingencies of course elections that accumulate into pathways is poorly understood and are almost always fateful. Ethnographic research suggests that students tend to select courses based on partial, poorly integrated information, in light of logistical constraints and personal preferences that have little directly to do with academics \cite{nathan2006my,rosenbaum2011complexities, rosenbaum2007after}. Absent conscientious design and signposting, students can easily spend time, credit hours, and tuition accumulating courses that do not lead efficiently to majors and completion \cite{bailey2015redesigning}. In addition to students, many other academic stakeholders could benefit from better information about college pathways. The work and priorities of instructors, department chairs, deans, and budget officers are implicated in the relative clarity of pathways and the efficiency with which courses can be sequenced to degree completion. 

%One challenge in mitigating this lack is that each of these
%potential beneficiaries of improvement require investigative equipment
%of different focal lengths.

%Day to day, {\bf academic advisors} on the ground lack tools for
%understanding the characteristics of the numerous majors and programs they are tasked to explain. For example, it is not trivial to identify majors whose requirements have students take courses primarily within one department. Requirement structures for majors such as Computer Science tend to impose this more intradepartmental regime. Other majors, such as History, instead have students consume courses such as economics and political science. Insight into such differences are needed for helping advisers suggest pathways that match their student clients' interests and temperaments.

%{\bf Instructors}, in contrast, require information local to their specific course offerings. Yet they cannot easily answer the question {\it Which majors, and preceding courses feed students into my class?}

%{\bf Department chairs} and {\bf deans} are for different reasons in need of knowing the migration paths of students in and out of departments. Questions such as {\it Does our school offer a rich set of non-major service courses to the rest of the university?}, and {\it are students from across campus uniformly taking advantage of such courses?} are difficult to answer.

%Beyond such specific needs, the myriad of possible pathways through curricula also impedes efforts by university administrators to encourage behaviors that lead to outcomes deemed desirable by their institutions. Such goals might emphasize persistence, intellectual breadth, or readiness for the workplace.

Fortunately the information necessary to observe pathways systematically and at scale is present in all colleges and universities in the form of academic transcripts. Transcripts are the official records documenting courses accumulated by each student as he or she makes academic progress. Yet transcripts typically are housed in tables of databases to which few have access, and in their ``raw'' form exceptionally opaque to interpretation. Most schools have specialized staff who generate specifically requested reports for individuals in high level academic positions\footnote{We acknowledge
here our own version of such a unit, which has provided us with numerous insights into both the information needs, and data semantics at our institution.}, but these personnel typically cannot interact directly with all the parties in need of insight on pathways at varying levels of detail.

%Some universities have created software that surfaces a selection of the needed information broadly to its institution's clients (e.g. \cite{carta}). But these tools focus primarily on the narrower needs of individual students, and therefore lack the ability to provide both overview and detailed insight.

We report here on {\it Via}, an analytic toolkit we have built to observe and understand undergraduate pathways utilizing de-identified transcript data held by a large private research university. Our approach is to provide a zoomable, investigative
instrument for large-scale, qualitative and quantitative
investigations of pathways. Built on graph theory, {\it Via} provides a
visual interface for observing the course sequences embedded in tens
of thousands of transcripts. In addition, {\it Via}'s grounding in graphs
allows us to bring associated mathematical computations to bear on the problem of pathway evolution.

Graph approaches have been applied to a wide variety of other tasks, such as detecting communities \cite{Fortunato2004}, collecting materials for survey articles \cite{ji2015}, the augmentation of collaborative recommendation records \cite{huang2005}, predicting future collaborations between scholars \cite{liben2007}, and suggesting drug interactions \cite{zitnik2018}. Our primary contribution in this paper is to apply graph approaches to the sequencing of academic coursework. Because our analytic strategy relies on data of a sort held by every legally recognized US college and university, it is amenable to application throughout the US postsecondary sector. 

%We explain how the availability of additional data, such as demographic information greatly adds to the results that can be achieved.

%\textit{Via} is a tool that enables education researchers to gain insights into how students structure their course decisions. One particularly useful feature of the model for this purpose is the ability to filter by the year a student enrolled in a particular course. This enables us to compare the dynamics of student course decisions across different years.

%has argued that the effect of growing introductory computer science courses has resulted largely as a result of the growing monetary returns that computer science fields offer. Educators fear, however, that the attraction to fields in computer and information science, however, may siphon away students interested in the humanities. Using our graph visualization of academic data we will illustrate that while we can observe an increase in the interest of student enrolling in computer science fields, a number of these courses also serve as popular courses that non-CS majors try out for fun. 


After related work we introduce {\em Via}, and examples of its
application.  We then explain the underlying computations that make
the visualizations possible.

%After related work, and a brief introduction to our dataset,
%Section~\ref{sec:visualization} introduces {\it Via}'s visual
%component. In Section~\ref{sec:mapping} we explain alternative graph
%mapping options, and their implications for questions that can be
%investigated. %Sections~\ref{sec:stud_matrix}/\ref{sec:aggr_matrix}
%detail the necessary processing of enrollment data in preparation to
%their ingestion into the visualization component.


