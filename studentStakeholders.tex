\subsection{Student Stakeholders}
\label{sec:student-stakeholders}

\begin{figure}
    \centering
    \includegraphics[width=0.55\columnwidth]{Figs/final-modularity-history.pdf}
    \includegraphics[width=0.44\columnwidth]{Figs/final-modularity-ee.pdf}
    \caption{A comparison between the student behavioral patterns in two departments---History (left) and Electrical Engineering (right)---from the years 2012-2016. Labels in the small circles are course names, such as ``\textsc{ee41}'' and ``\textsc{history198}''.}
    \label{fig:modularity}
\end{figure}

We next highlight two particular use cases. First, our visual modeling
of course sequences allows a student to identify which majors require
students to take courses from within primarily a certain department,
and which allow for more academic exploration across departments. We
can report this result both visually, by directly looking at the
intra-department orange arrows in a graph, and by calculating, the
{\em modularity} of certain departments. Modularity is another example
of how {\em Via}'s choice of graphs as building blocks enables the
application of well known mathematical tools. The metric represents
the connectivity of clusters within a graph, by calculating the
over-representation of edges among groups of nodes.

In Figure~\ref{fig:modularity}, we compare the interconnectivity of
the History, and Electric Engineering majors, limited to course
enrollments of one academic year apart.  Here we see that the
visualizations align with the modularity scores associated with
History (0.003) and Electrical Engineering (0.028). The higher the
modularity score the more a department is intra-connected. This effect
is caused by highly prescriptive requirement structures.

As a second use case, we illustrate how \textit{Via} can be leveraged
to discover the ''try-me'' courses. Departments offer such service courses for students
majoring in unrelated fields, but who are interested in exploring other
areas of study. {\em Via} enables quick discovery of such courses.

We find the ``try-me'' courses by creating a {\em Via} graph over
students of a single major. We then identify the most popular courses
within this generated graph that are not in that major. For instance,
by filtering on the History major, we observe that of the History
majors, nearly 28\% take \textsc{cs105} and 21\% take
\textsc{cs106a}. These trends are visualized through variations in
node coloration in Figure~\ref{fig:history-try-me}.

\begin{figure}[h]
    \centering
    \includegraphics[width=.9\columnwidth]{Figs/final-history-try-me.pdf}
    \caption{A visualization of the most popular courses in the Computer Science department for History majors.}
    \label{fig:history-try-me}
\end{figure}

 Babad et al. have discovered that the primary reasons students decide
 to enroll in a particular course are the learning value of the class
 followed closely by the lecture style \cite{Babad2003}. Particularly
 difficult courses are avoided by students unless otherwise
 required. Our assessment of the recommended Computer Science
 ``try-me'' courses for History majors corroborates these
 results. Although \textsc{cs106a} is branded as the most enrolled
 introductory Computer Science course at the university from which we
 have acquired our dataset, \textsc{cs105} additionally presents
 itself as the Computer Science course for non-majors, and it is also
 known among students as less rigorous. Our \textit{Via} toolkit is
 thus able to offer a more personalized course discovery system for
 students of a particular major.
 
