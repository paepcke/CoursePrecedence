\subsection{Course Discovery}
An important use case of our model is predicting prerequisites for
courses which have no explicitly listed required classes. Because {\em
  Via} models the interactions between courses as a graph, we can
leverage the {\em PageRank} algorithm for this task
\cite{Page1999}. PageRank measures the relative importance between nodes in a graph by conducting random walks through a network.

The algorithm is best known for its application in Web search. In the
curriculum domain, PageRank models many generations of students moving
through courses. The model provides probabilities of courses that
students with a given course history will take subsequently in their
college careers: it illuminates pathways based on prior
behavior. The random walk in this context is an imaginary new student
who is enrolled in a {\em seed course}. From there the student
randomly enrolls in successor courses, guided by the probabilities in
the projection matrix, until the student reaches a pre-determined
target course $C$.

Importantly, PageRank goes beyond counting enrollments---the course
node incidence of links into $C$. The algorithm mirrors the
probability that a student arrives at $C$ from other courses, which
might be topically distant. The algorithm thus takes into account path
lengths, not just the immediate enrollment history prior to $C$.

We demonstrate two examples of how the PageRank algorithm grants us
insights into implicit prerequisite relationships that exist between
courses.

{\bf Which math alternative to choose:} For the first example, let us imagine a student who hopes to take \textsc{cs229}, a course that covers a broad introduction to Machine Learning methods.  The course is offered by
the Computer Science department in the School of Engineering.

In order to take this class, students are required to understand
matrix calculus and linear algebra. The two introductory mathematics
courses offered to first-year undergraduates are \textsc{cme100} and
\textsc{math51}. Among these two courses, \textsc{cme100} is offered by the School of Engineering, and is branded as an introductory mathematics course with an engineering focus. Based on the course catalog alone, \textsc{cme100} is ostensibly the introductory math course to take if one is interested in engineering. \textsc{math51} is offered as the more general introductory math course, often taken by students from outside the School of Engineering. Taking \textsc{cme100} over \textsc{math51} seems like the obvious choice for a student seeking to take \textsc{cs229}. 

Yet aggregated data on student behavior suggests otherwise. Running PageRank twice, once with the seed course set as \textsc{math51} and once with \textsc{cme100}, we find that a student is more likely to complete
\textsc{cs229} after completing \textsc{math51}. Contrary to our premonition, this result indicates a stronger sequential relationship
between \textsc{math51} and \textsc{cs229} than between \textsc{cme100} and \textsc{cs229}. We suspect that students, instructors, and administrators all might benefit from illuminations of this sort.

One might argue that such insights could be gleaned from simple
database queries over enrollment data to show that \textsc{math51} and \textsc{cs229} have a higher joint enrollment count than \textsc{cme100} and \textsc{cs229}. When analyses are even mildly more complex than that, however, the approach becomes intractable.

Consider the case in which we are conditioning on more than
one course---say a student's entire course history. The intersection
of students with this precise set of courses is small, perhaps too
small to derive any meaningful statistical analysis of future
behavior. In its use of random walks, PageRank gives us a simple, yet
efficient method of approximating common student behaviors given a set
of courses already taken. The following illustrates this point.

