\subsection{Course Discovery}
An important use case of our model is predicting prerequisites for
courses which have no explicitly listed required classes. Because {\em
  Via} models the interactions between courses as a graph, we can
leverage the {\em PageRank} algorithm for this task
\cite{Page1999}. PageRank is an algorithm that measures the relative
importance between nodes in a graph by conducting random walks through
a network.

The algorithm is best known for its application in Web search. In the
curriculum domain PageRank models many generations of students moving
through courses. The model provides probabilities about courses
students with a given course history will take later on in their
careers; it illuminates student pathways based on past student
behavior. The random walk in this context is an imaginary new student
who is enrolled in a {\em seed course}. From there the student
randomly enrolls in successor courses, guided by the probabilities in
the projection matrix, until the student reaches a pre-determined
target course $C$.

Importantly, PageRank goes beyond counting enrollments---the course
node incidence of links into $C$. The algorithm mirrors the
probability that a student arrives at $C$ from other courses, which
might be topically distant. The algorithm thus takes into account path
lengths, not just the immediate enrollment history prior to $C$.

We demonstrate two examples of how the PageRank algorithm grants us
insights into implicit prerequisite relationships that exist between
courses.

{\bf Which math alternative to choose:} For the first example, let us
image a student who hopes to take CS229, a course that covers a broad
introduction to Machine Learning methods.  The course is offered by
the Computer Science department in the School of Engineering.

In order to take this class, students are required to understand
matrix calculus and linear algebra. The two introductory mathematics
courses offered to first-year undergraduates are CME100 and
MATH51. Among these two courses, CME100 is offered by the School of
Engineering, and is branded as an introductory mathematics course with
an engineering focus. Based on the course catalog alone CME100 is thus
ostensibly the introductory math course to take if interested in
engineering. Math51 is offered as the more general introductory math
course, often taken by students from outside the School of
Engineering. Taking CME100 over MATH51 seems like the obvious choice
for a student seeking to take CS229. Yet, aggregated student behavior
suggests otherwise.

Running PageRank twice, once with MATH51 set as the seed course and
once with CME100, we find that a student is more likely to complete
CS229 after completing MATH51. This result indicates, contrary to our
initial impression, that a stronger sequential relationship exists
between MATH51 and CS229 than between CME100 and CS229. When making
decisions about which courses to take, students can benefit from these
additional quantitative insights to supplement the information that
simple course descriptions provide.

One could argue that such an analysis can be replicated with a simple
database queries over enrollment data to show that MATH51 and CS229
have a higher joint enrollment count than CME100 and CS229. However,
when analyses are even mildly more complex, this approach grows
intractable.

Consider the case in which we are conditioning on more than
one course---say a student's entire course history. The intersection
of students with this precise set of courses is small, perhaps too
small to derive any meaningful statistical analysis of future
behavior. PageRank, in its use of random walks, gives us a simple, yet
efficient method of approximating common student behaviors given a set
of courses already taken. The following illustrates this point.

