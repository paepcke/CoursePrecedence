\section{Advanced Applications}
\label{sec:advanced}

With the underlying data structures explained, it is now clear how
{\em Via} builds special subgraphs by filtering early, when the
student level information is still available. The filtering occurs as
matrix $M$ is constructed. Similarly, the discount function is used to
control how closely together we wish enrollments to have occurred in
time. Several of the applications above used these techniques.

We next introduce applications that rely on the graph data structure
less as a visualization method. Instead curriculum insights are gained
from mathematical analyses of the graph that is described by the
projection matrix $P$.\footnote{At the time of this writing we have not
  yet provided access to these methods through {\em Via}'s user
  interface.}




