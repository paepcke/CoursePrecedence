{\bf De facto Prerequisites:} Consider the common scenario in
which a first-year undergraduate finds an upper
level course with few or no listed prerequisites. In the following
example the course is \textsc{polisci1d}, {\em Democracy, Development, and
  the Rule of Law}. Its student composition skews heavily towards
third- and fourth- year undergraduates and graduate students. The
student wishes to find a course that will prepare her for this upper
level offering.

We seed the PageRank algorithm with the four courses that the student
has already taken. We set \textsc{polisci1d} to be the target of all random
walks (enrollments).

\begin{figure}
    \centering
    \noindent\fbox{%
    \parbox{\columnwidth}{%
        \textbf{POLISCI~1L: Modern Political Thought: Machiavelli to Marx and Mill}
        This course offers an introduction to the history of Western political thought from the late fifteenth through the nineteenth centuries. We will consider the development of ideas like individual rights, government by consent, and the protection of private property. We will also explore the ways in which these ideas continue to animate contemporary political debates. Thinkers covered will include: Niccolò Machiavelli, Thomas Hobbes, John Locke, Jean-Jacques Rousseau, Edmund Burke, John Stuart Mill, and Karl Marx.
        }%
    }
    \noindent\fbox{%
    \parbox{\columnwidth}{\textbf{POLISCI~1D: Democracy,
        Development, and the Rule of Law} This course explores the
      different dimensions of development---economic, social, and
      political---as well as the way that modern institutions (the
      state, rule of law, and democratic accountability) developed and
      interacted with other factors across different societies around
      the world.  } }
    \caption{Descriptions of the top intermediary course candidate,
      POLISCI~1L, and the target course, POLISCI~1D.}
    \label{fig:pr-course-descriptions}
\end{figure}

We then iterate through the course catalog: at each iteration, we take
one course as the candidate for best preparatory course. We append it
to the initial four-class seed, 'pretending' the student had taken
this course. We then run PageRank from the augmented seed set, and
note the PageRank score of the target course. Upon loop termination,
we select as good candidates the top $k$ courses that led to the
highest target course PageRank.  See
Figure~\ref{fig:course-recommendation-algorithm} for a pseudo-code
implementation of the algorithm for $k=1$.

In this example, the top candidate course discovered by our algorithm
is \textsc{polisci1l}, {\em Modern Political Thought: Machiavelli to Marx
  and Mill}. The catalog excerpts in Figure
\ref{fig:pr-course-descriptions} confirm that this course is a good
choice. Unlike the target course, 71\% of students take the course in
their first two years. This example shows how PageRank reaches beyond
analyses derived solely from enrollment data queries.

%% We can immediately see a relationship between the two
%% courses in their official catalog descriptions (full texts found in
%% Figure \ref{fig:pr-course-descriptions}): the candidate course focuses
%% on foundational concepts such as ``individual rights, government by
%% consent, and the protection of private property,'' and the target
%% course focuses on ``the way that modern institutions (the state, rule
%% of law, and democratic accountability) developed and interacted with
%% other factors across different societies around the world.''

\lstset{language=Python}          % Set your language (you can change the language for each code-block optionally)

\begin{figure}
    \begin{lstlisting}[frame=single] 
let max_score = 0
let best_course = None 

let initial_classes = [ 'POLISCI1',
                        'STATS1',
                        'POLISCI2',
                        'POLISCI3']
                        
let target_class = 'POLISCI1D'

for course in courses:
    if course is in initial_classes:
        continue
    else:
        initial_classes.append(course)
        score = page_rank(initial_classes
                          target_class)
        if score > max_score:
            max_score = score
            best_course = course
        
        initial_classes.remove(course)

return (max_score, best_course)
\end{lstlisting}
\caption{Pseudo-code implementation of a course candidate search algorithm based on PageRank scores.}
\label{fig:course-recommendation-algorithm}
\end{figure}

The algorithm features several tunable parameters. For example, a
student can weigh their seed set such that some courses have higher
influence over the final decision than others. A student who may have
recently switched majors may want their queries to skew more heavily
towards their new major. In these cases, a student may decide to weigh
their most recent courses more heavily than others. Students can also
constrain the candidate course set to courses they have already
researched, as opposed to performing a search over the entire course
catalog.
