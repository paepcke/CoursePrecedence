\documentclass{sigchi}

% Use this section to set the ACM copyright statement (e.g. for
% preprints).  Consult the conference website for the camera-ready
% copyright statement.

% Copyright
\CopyrightYear{2016}
%\setcopyright{acmcopyright}
\setcopyright{acmlicensed}
%\setcopyright{rightsretained}
%\setcopyright{usgov}
%\setcopyright{usgovmixed}
%\setcopyright{cagov}
%\setcopyright{cagovmixed}
% DOI
\doi{http://dx.doi.org/10.475/123_4}
% ISBN
\isbn{123-4567-24-567/08/06}
%Conference
\conferenceinfo{CHI'16,}{May 07--12, 2016, San Jose, CA, USA}
%Price
\acmPrice{\$15.00}

% Use this command to override the default ACM copyright statement
% (e.g. for preprints).  Consult the conference website for the
% camera-ready copyright statement.

%% HOW TO OVERRIDE THE DEFAULT COPYRIGHT STRIP --
%% Please note you need to make sure the copy for your specific
%% license is used here!
% \toappear{
% Permission to make digital or hard copies of all or part of this work
% for personal or classroom use is granted without fee provided that
% copies are not made or distributed for profit or commercial advantage
% and that copies bear this notice and the full citation on the first
% page. Copyrights for components of this work owned by others than ACM
% must be honored. Abstracting with credit is permitted. To copy
% otherwise, or republish, to post on servers or to redistribute to
% lists, requires prior specific permission and/or a fee. Request
% permissions from \href{mailto:Permissions@acm.org}{Permissions@acm.org}. \\
% \emph{CHI '16},  May 07--12, 2016, San Jose, CA, USA \\
% ACM xxx-x-xxxx-xxxx-x/xx/xx\ldots \$15.00 \\
% DOI: \url{http://dx.doi.org/xx.xxxx/xxxxxxx.xxxxxxx}
% }

% Arabic page numbers for submission.  Remove this line to eliminate
% page numbers for the camera ready copy
% \pagenumbering{arabic}

% Load basic packages
\usepackage{balance}       % to better equalize the last page
\usepackage{graphics}      % for EPS, load graphicx instead 
\usepackage[T1]{fontenc}   % for umlauts and other diaeresis
\usepackage{txfonts}
\usepackage{mathptmx}
\usepackage[pdflang={en-US},pdftex]{hyperref}
\usepackage{color}
\usepackage{booktabs}
\usepackage{textcomp}

% Some optional stuff you might like/need.
\usepackage{microtype}        % Improved Tracking and Kerning
% \usepackage[all]{hypcap}    % Fixes bug in hyperref caption linking
\usepackage{ccicons}          % Cite your images correctly!
% \usepackage[utf8]{inputenc} % for a UTF8 editor only

% If you want to use todo notes, marginpars etc. during creation of
% your draft document, you have to enable the "chi_draft" option for
% the document class. To do this, change the very first line to:
% "\documentclass[chi_draft]{sigchi}". You can then place todo notes
% by using the "\todo{...}"  command. Make sure to disable the draft
% option again before submitting your final document.
\usepackage{todonotes}

% Paper metadata (use plain text, for PDF inclusion and later
% re-using, if desired).  Use \emtpyauthor when submitting for review
% so you remain anonymous.
\def\plaintitle{Via: Illuminating Course Choice and Discipline Interactions}
\def\plainauthor{First Author, Second Author, Third Author,
  Fourth Author, Fifth Author, Sixth Author}
\def\emptyauthor{}
\def\plainkeywords{Authors' choice; of terms; separated; by
  semicolons; include commas, within terms only; required.}
\def\plaingeneralterms{Documentation, Standardization}

% llt: Define a global style for URLs, rather that the default one
\makeatletter
\def\url@leostyle{%
  \@ifundefined{selectfont}{
    \def\UrlFont{\sf}
  }{
    \def\UrlFont{\small\bf\ttfamily}
  }}
\makeatother
\urlstyle{leo}

% To make various LaTeX processors do the right thing with page size.
\def\pprw{8.5in}
\def\pprh{11in}
\special{papersize=\pprw,\pprh}
\setlength{\paperwidth}{\pprw}
\setlength{\paperheight}{\pprh}
\setlength{\pdfpagewidth}{\pprw}
\setlength{\pdfpageheight}{\pprh}

% Make sure hyperref comes last of your loaded packages, to give it a
% fighting chance of not being over-written, since its job is to
% redefine many LaTeX commands.
\definecolor{linkColor}{RGB}{6,125,233}
\hypersetup{%
  pdftitle={\plaintitle},
% Use \plainauthor for final version.
%  pdfauthor={\plainauthor},
  pdfauthor={\emptyauthor},
  pdfkeywords={\plainkeywords},
  pdfdisplaydoctitle=true, % For Accessibility
  bookmarksnumbered,
  pdfstartview={FitH},
  colorlinks,
  citecolor=black,
  filecolor=black,
  linkcolor=black,
  urlcolor=linkColor,
  breaklinks=true,
  hypertexnames=false
}

% create a shortcut to typeset table headings
% \newcommand\tabhead[1]{\small\textbf{#1}}

% End of preamble. Here it comes the document.
\begin{document}

\title{\plaintitle}

\numberofauthors{3}
\author{%
  \alignauthor{Leave Authors Anonymous\\
    \affaddr{for Submission}\\
    \affaddr{City, Country}\\
    \email{e-mail address}}\\
  \alignauthor{Leave Authors Anonymous\\
    \affaddr{for Submission}\\
    \affaddr{City, Country}\\
    \email{e-mail address}}\\
  \alignauthor{Leave Authors Anonymous\\
    \affaddr{for Submission}\\
    \affaddr{City, Country}\\
    \email{e-mail address}}\\
}

\maketitle

\begin{abstract}
Graph theoretic approaches can enable concise observation of how students navigate curricular choices in higher education. We exemplify the approach using 18 years of data describing course enrollments of undergraduates at a large research university. After showing options for how student movement through courses can be mapped onto graphs, we model courses as nodes, and sequential course taking as links. Subsequent examples highlight differences in departmental course sequences, both visually and via modularity computation. We demonstrate the quantification of university wide service course usage with graph based visual and computational approaches. We illustrate the value of graph theoretic approaches for learning scientists, academic administrators, and students.
 \end{abstract}

\category{H.5.m.}{Information Interfaces and Presentation
  (e.g. HCI)}{Miscellaneous} \category{See
  \url{http://acm.org/about/class/1998/} for the full list of ACM
  classifiers. This section is required.}{}{}

\keywords{\plainkeywords}

\section{Introduction}
%%{\color{red}[LOCKED BY ANDREAS]}


%% - What is the problem?
%% - Why is it important?
%% - Why is it hard?
%% - How are current solutions insufficient?
%% - What we do

US higher education is unique in the world in the extent to
which schools expect undergraduates to explore a variety of courses before committing to a field of study. In contrast with virtually all other national postsecondary systems, in which students enter schools and programs with relatively structured curricula, US undergraduates are encouraged to explore a variety of academic options through an iterative course search and selection process. We call a student's
eventual sequence of course elections a {\it pathway}. Pathways are notoriously poorly instrumented for observation by students, educators, administrators or researchers \cite{chambliss2014college}.

While enriching when successful, the contingencies of course elections that accumulate into pathways is poorly understood and are almost always fateful. Ethnographic research suggests that students tend to select courses based on partial, poorly integrated information, in light of logistical constraints and personal preferences that have little directly to do with academics \cite{nathan2006my,rosenbaum2011complexities, rosenbaum2007after}. Absent conscientious design and signposting, students can easily spend time, credit hours, and tuition accumulating courses that do not lead efficiently to majors and completion \cite{bailey2015redesigning}. In addition to students, many other academic stakeholders could benefit from better information about college pathways. The work and priorities of instructors, department chairs, deans, and budget officers are implicated in the relative clarity of pathways and the efficiency with which courses can be sequenced to degree completion. 

%One challenge in mitigating this lack is that each of these
%potential beneficiaries of improvement require investigative equipment
%of different focal lengths.

%Day to day, {\bf academic advisors} on the ground lack tools for
%understanding the characteristics of the numerous majors and programs they are tasked to explain. For example, it is not trivial to identify majors whose requirements have students take courses primarily within one department. Requirement structures for majors such as Computer Science tend to impose this more intradepartmental regime. Other majors, such as History, instead have students consume courses such as economics and political science. Insight into such differences are needed for helping advisers suggest pathways that match their student clients' interests and temperaments.

%{\bf Instructors}, in contrast, require information local to their specific course offerings. Yet they cannot easily answer the question {\it Which majors, and preceding courses feed students into my class?}

%{\bf Department chairs} and {\bf deans} are for different reasons in need of knowing the migration paths of students in and out of departments. Questions such as {\it Does our school offer a rich set of non-major service courses to the rest of the university?}, and {\it are students from across campus uniformly taking advantage of such courses?} are difficult to answer.

%Beyond such specific needs, the myriad of possible pathways through curricula also impedes efforts by university administrators to encourage behaviors that lead to outcomes deemed desirable by their institutions. Such goals might emphasize persistence, intellectual breadth, or readiness for the workplace.

Fortunately the information necessary to observe pathways systematically and at scale is present in all colleges and universities in the form of academic transcripts. Transcripts are the official records documenting courses accumulated by each student as he or she makes academic progress. Yet transcripts typically are housed in tables of databases to which few have access, and in their ``raw'' form exceptionally opaque to interpretation. Most schools have specialized staff who generate specifically requested reports for individuals in high level academic positions\footnote{We acknowledge
here our own version of such a unit, which has provided us with numerous insights into both the information needs, and data semantics at our institution.}, but these personnel typically cannot interact directly with all the parties in need of insight on pathways at varying levels of detail.

%Some universities have created software that surfaces a selection of the needed information broadly to its institution's clients (e.g. \cite{carta}). But these tools focus primarily on the narrower needs of individual students, and therefore lack the ability to provide both overview and detailed insight.

We report here on {\it Via}, an analytic toolkit we have built to observe and understand undergraduate pathways utilizing de-identified transcript data held by a large private research university. Our approach is to provide a zoomable, investigative
instrument for large-scale, qualitative and quantitative
investigations of pathways. Built on graph theory, {\it Via} provides a
visual interface for observing the course sequences embedded in tens
of thousands of transcripts. In addition, {\it Via}'s grounding in graphs
allows us to bring associated mathematical computations to bear on the problem of pathway evolution.

Graph approaches have been applied to a wide variety of other tasks, such as detecting communities \cite{Fortunato2004}, collecting materials for survey articles \cite{ji2015}, the augmentation of collaborative recommendation records \cite{huang2005}, predicting future collaborations between scholars \cite{liben2007}, and suggesting drug interactions \cite{zitnik2018}. Our primary contribution in this paper is to apply graph approaches to the sequencing of academic coursework. Because our analytic strategy relies on data of a sort held by every legally recognized US college and university, it is amenable to application throughout the US postsecondary sector. 

%We explain how the availability of additional data, such as demographic information greatly adds to the results that can be achieved.

In our first approximation here, we convert transcript information describing course enrollments at our case school to directed graphs, with nodes modeling courses, links modeling sequential enrollments, and link weights modeling conditional probabilities of enrolling in a particular course before another one. We partition the graphs by the departments to which each course is officially associated by the university registrar. An existing graphing tool \cite{shannon2003cytoscape} exposes more or less information, depending on chosen zoom levels.

%While the concept is easy to explain, details arise that are not
%immediately obvious. In addition, the most informative mapping from
%enrollments to graphs depends on the questions the graph is intended
%to answer. Resulting mappings can vary widely.

After related work and a brief introduction to our dataset, Section~\ref{sec:methodology} introduces how we construct the node and
edge relationships in the {\em Via} network. Section~\ref{sec:visualization} then demonstrates the visual component of the model.  Section~\ref{sec:analysis} highlights diverse use-cases of our model to address questions of different academic stakeholders: students, instructors, and administrators. This final section demonstrates the wider applicability of our model, by comparing our results against well-studied phenomena in education research.

%\textit{Via} is a tool that enables education researchers to gain insights into how students structure their course decisions. One particularly useful feature of the model for this purpose is the ability to filter by the year a student enrolled in a particular course. This enables us to compare the dynamics of student course decisions across different years.

%has argued that the effect of growing introductory computer science courses has resulted largely as a result of the growing monetary returns that computer science fields offer. Educators fear, however, that the attraction to fields in computer and information science, however, may siphon away students interested in the humanities. Using our graph visualization of academic data we will illustrate that while we can observe an increase in the interest of student enrolling in computer science fields, a number of these courses also serve as popular courses that non-CS majors try out for fun. 


%After related work, and a brief introduction to our dataset,
%Section~\ref{sec:visualization} introduces {\it Via}'s visual
%component. In Section~\ref{sec:mapping} we explain alternative graph
%mapping options, and their implications for questions that can be
%investigated. %Sections~\ref{sec:stud_matrix}/\ref{sec:aggr_matrix}
%detail the necessary processing of enrollment data in preparation to
%their ingestion into the visualization component.





\section{Related Work}




\section{Data}
The dataset we use to build our graph for mapping sequential relationships between classes was obtained directly from the Stanford Carta Lab. This dataset contains anonymized class enrollment data for over 52,000 students (both graduates and undergraduates) who were enrolled at Stanford during any time between Fall 2000 and Fall 2018. Each of the over two million entries in this table includes a unique hash which corresponds to each student, the course in which they enrolled, and the quarter during which they enrolled. Our dataset also contains supplementary information on a student's major during time of enrollment in a course and each student's major upon graduation. Depending on the class of problems of interest, we filter along these additional values.

\subsection{Data Preprocessing}

Data preprocessing involved the parsing of the requisite information made available through the Carta dataset; for each student, all classes are paired with their quarters of enrollment. Student metadata was omitted and only classes in our dataset marked completed (classes not dropped by the student during the quarter) were included.

% \begin{figure}[h!]
% \begin{tcolorbox}
% Abstraction and its relation to programming. Software engineering principles of data abstraction and modularity. Object-oriented pro
% gramming, fundamental data structures (such as stacks, queues, sets) and data-directed design. Recursion and recursive data structures (linked lists, trees, graphs). Introduction to time and space complexity analysis. Uses the programming language C++ covering its
% basic facilities. Prerequisite: 106A or equivalent. Summer quarter enrollment is limited.
% \end{tcolorbox}
% \caption{Course Description for CS106B: Programming Abstraction.}
% \label{course-description}
% \end{figure}

\section{Methodology}

Our principal objective was to build a powerful, yet interpretable projection model capable of visualizing the sequential relationship of course pairs given a school's enrollment data. We believe this model can serve as a toolkit for university stakeholders, allowing them to make more informed decisions about courses. Our graphical framework is built on a flexible projection algorithm that enables the user to adapt the model's visualizations to answer a variety of questions. 

There are two components to the generation of a projection model. The first involves the preparation of a sequence matrix--- the second involves the calculation of the projection itself.

\subsection{Sequence Matrix Generation}

The Via toolkit receives data that represent student enrollment in courses. The projection algorithm must take a matrix $A$ of shape $(|S|, |C|)$ as input, where $S$ is the set of all students and $C$ is the set of all courses. We can interpret this matrix $A$ as a bipartite multigraph documenting the interaction of student nodes with courses nodes over time. A given entry $A_{ij}$ represents the relative timestep a student $i$ enrolls in some course $j$. For example, an entry $A_{ij} = 1$ signifies that student $i$ enrolled in course $j$ during their first quarter at Stanford. An entry $A_{ij} = 0$ implies that $i$ never enrolled in $j$. This thus implies that each row $A_i$ represents the entire course enrollment history of some student $i$. We generate various forms of matrix $A$ from the raw enrollment data by filtering on different student attributes in order to gain perspectives on different aspects of enrollment behavior. The attribute set we leverage in this analysis are mainly a student's major upon enrollment in a course, a student's final major, and the year in which a course was taken. 

\subsection{Graph Projection}

Given sequence matrix $A$, we generate a final matrix $P$ of shape $(|C|, |C|)$ that can be interpreted as an adjacency matrix for a one-mode projection of the bipartite network described above. The edges in our projection represent a set of conditional probabilities defined between pairs of courses. The parameters for these conditional probabilities are fit based on counts determined in the calculation of intermediary matrix $\tilde{P}$ of shape $(|C|, |C|)$ . 

We calculate each entry $\tilde{P}_{ij}$ by accumulating the occurrences of course $i$ taken at some point before course $j$ across all students in set $S$:
\begin{equation}
  \tilde{P}_{ij} = \sum_{s=1}^{|S|} \mathbbm{1}\{A_{si} - A_{sj} \geq 0\}* d(A_{si} - A_{sj})
\end{equation}
where $A_{si} - A_{sj}$ can be interpreted as the academic timestep delta (commonly measured in semesters, or quarters) between a student $s$'s taking course $i$ and course $j$, and $d$ is a function chosen to modulate the signal of the event given the timestep delta if desired. 

The function $d$ may be continuous or discrete.  For example, the
following choice attenuates the relationship between two courses through an
exponential decay:
\begin{equation}
  \tilde{P}_{ij} = \sum_{s=1}^{|S|} \mathbbm{1}\{A_{si} - A_{sj} \geq 0\}* \lambda^{A_{si} - A_{sj}}
\end{equation}
where, $\lambda$ is a constant that controls decay rate. Alternatively, the following choice for $d$ counts a relationship between two courses only when course $j$ immediately follows course $i$. Any
elapsed time in between the two courses severs the relationship:
\begin{equation}
d = \begin{cases} 
      1, & A_{si} - A_{sj} = 1 \\
      0, & A_{si} - A_{sj} > 1 
    \end{cases}
\end{equation}

With $\tilde{P}$, we calculate the final projection matrix $P$. Entry $P_{ij}$ is a conditional probability $p(j|i)$ whose parameters can be computed with the following closed form expression:
\begin{equation}
    P_{ij} = p(j|i) = \frac{\tilde{P}_{ij}}{\sum_{s=1}^{|S|}\mathbbm{1}\{A_{si} > 0\}}
\end{equation}
$P_{ij}$ thus represents the proportion of students who take the course sequence: course $i$ followed by $j$ out of the total number of students who take course $i$ at any point. One may note that this bears a resemblance to a Bayesian Network; however, we make the assumption that the process of transitioning from course node to course node is Markovian in nature. This eases implementation but precludes our model from being a true Bayesian Network due to the emergence of cycles.

Finally, observe that the selection of timestep delta function $d$ greatly influences the relationships expressed by the final projection model. This is by design in order to allow for greater flexibility in the search for solutions to various problem classes. We present results using a discrete function in order to maximize model interpretability.

\section{Graph Analysis} 


\section{Visualization and Interaction}
\label{sec:visualization}

\section{Mapping Enrollments to Graphs}
\label{sec:mapping}

\section{Student Level Pathway Matrix}\
\label{sec:stud_matrix}

\section{Aggregate Path Matrix}
\label{sec:aggr_matrix}

%% \section{Other Applications}

\section{Conclusion and Future Work}
\label{sec:conclusion}
We have demonstrated a new approach to visualizing and interpreting transcript enrollment data available at every major American university. We've proposed a visualization toolkit that allows stakeholders such as students, advisors, and department chairs to understand aggregate student behavior over time. Using graph theoretic principles, we have presented several use cases for our projection model and in doing so have provided some insight into course trends at Stanford University.

We will be continuing experiments using this paradigm in order to better understand the effects of the demand increase in Computer Science at a university wide level. We have just scratched the surface in student behavior analysis. Future work includes running various feature representation algorithms such as node2vec and DeepWalk in order to visualize changes in course relationships over time. We will also be analyzing the results of running the PageRank algorithm on the various projection models made possible by our model in an attempt to better understand the similarities between courses.

% BALANCE COLUMNS
\balance{}

% REFERENCES FORMAT
% References must be the same font size as other body text.
\bibliographystyle{SIGCHI-Reference-Format}
\bibliography{coursePrecedence}

\end{document}

%%% Local Variables:
%%% mode: latex
%%% TeX-master: t
%%% End:
