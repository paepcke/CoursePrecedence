\section{Conclusion and Future Work}
\label{sec:conclusion}
We have described and demonstrated our {\em Via} toolkit, which is
constructed atop a new approach for visualizing and interpreting
transcript enrollment data. Such data are available at every major
US university. {\em Via} allows stakeholders as diverse as
students, instructors, and administrators to understand aggregate
student behavior over time. Using graph visualization and graph
theoretic computation in concert we have presented several {\em Via}
use cases.

We explained the process through which standard enrollment data can be
transformed into graph structures that may be tuned to particular
investigative goals. This flexibility arises both during graph
construction, and during interactive manipulation of the
graphs. We deploy the existing tool Cytoscape for such
manipulations, but other tools may be just as appropriate, once graphs
are constructed through the algorithm we have presented.

We will continue our exploration by investigating networks with
multiple node types: one to represent students, another to model
courses. The answers to other types of questions will be found through
this different family of graphs. 

We further plan to extend {\em Via} to include support for simulations
and what-if analyses. Further effort will also need to be invested
into making the graph construction process easy to use. 

Many questions may be answered by skillful SQL queries over university
datasets. But these approaches often fall short when questions are
not yet clearly defined, and relatively large amounts of data need to
be shaped and reshaped to discover patterns. As postsecondary
education is increasingly held accountable for performance, a deep
understanding of such patterns and trends over time will be
required. We have presented a fresh step toward such capability.
