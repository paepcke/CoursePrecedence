\section{Related Work}

%A number of scholars have investigated the dilemma of rich curricular
%choice being desirable, while the plethora of options can at the same
%time lead to dropout, and other detrimental outcomes for students.
%For example, the dissertation \cite{slim2016} asserts that ``there is
%an inverse correlation between the complexity of a curriculum and the
%graduation rate of students attempting that curriculum.''

The dangers of choice overload are well documented. Research in marketing \cite{botti2006} and psychology \cite{schwartz2004paradox,iyengar2000} has demonstrated both the stated preference for
choice, and improved factual success when choice is limited. In higher education, Bake et al. \cite{bake2018} find that simply reducing choice is not an acceptable answer. While students do not have strong reactions to increased guidance, they react negatively to a reduction in choice. Thus enriching guidance is one promising
approach. The author of \cite{slim2016} shows that simulations and
predictions based on student models can successfully identify points
where advisors could intervene to improve graduation outcomes.

Community colleges and other broad-access schools are at least as affected by the tension between the benefits of choice and student confusion. Educational
outcomes in community colleges are negatively affected by ``chaotic
enrollment patterns'' \cite{crosta2014, bail2015, scot2015}. A number of these institutions have explored options for improved support by providing ``guided pathways'' \cite{jenkins2013}, but without the benefit of graph visualization and other computational-techniques. A few comprehensive universities have built carefully instrumented ``early alert'' and other advising tools utilizing institutional data \cite{fletcher2016integrated}, but the computational techniques underlie these tools remain opaque. 

Throughout both the social and physical sciences, graphs are used to visualize, simplify and facilitate computational analysis of complex dynamic systems. Graphs have been applied to model a diverse set of networks such as food chains \cite{Hall1993}, the human genome \cite{Pevzner1989} and ecological systems \cite{Fortin2012}. Within the social sciences, graphing methods have a long history of application in the study of systems-level phenomena \cite{Borgatti2009}. The use of graphs within the social sciences was particularly spurred on by the insight that human societies could be structured like biological systems. The $19^{th}$ century French philosopher, Emile Durkheim, argued for instance that social regularities could be found in the structure of social environments in which they were embedded \cite{Durkheim1951}. By studying systemic regularities it is possible to derive macro-level insights about the structure of many micro-level interactions. 

Since the mid-1950s, graphs have been applied to model the flow of information in social and professional networks. As part of the MIT Group Networks Laboratory, Leavitt et al. observed how the structure of interpersonal relationships between groups of coworkers facilitated the spread of information throughout a team of colleagues \cite{Leavitt1951}. More recently, similar methods have been applied to study how workplace professional networks influence the spread of information through company email chains \cite{Fisher2004}. The modeling of knowledge transmission parallels closely how graph-theoretic methods have been applied to understand social dynamics in the field of education. Studies of citation networks, for instance, describe how intellectual advances spread through academic space \cite{Batagelj2003, Garfield1964}. 


% If individual publications are modeled as nodes in the citation network and edges represent a direct citation of one paper by another it should not be feasible for cycles to exist in the citation network. A similar logic applies to the structuring of course sequences in which cycles are not desirable among introductory classes that are intended to be taken sequentially.

A number of the challenges posed in representing citation networks, such as learning optimal edge weights, are directly applicable to the problem of modeling course sequences. Within citation networks, it is useful to modulate the edge strength between two papers in order to represent the relative influence of a cited paper. Batagelj proposes solutions to this problem by introducing SPC weights on each edge of the network to capture the incoming and outgoing ``flow of information'' for a given paper \cite{Batagelj2003}. Hajra et al. also observe aging phenomena among papers in which the probability that a paper is cited decays with time at an exponential rate proportional to $t^{0.9}$ up to ten years after its publication date \cite{Hajra2005}. Course selection may display a similar exponential time-dependent probability. Moreover, just as citation networks provide a framework for modeling the flow of knowledge within academia, course sequence networks imply shared and prerequisite knowledge between courses. 

Also analogous to our particular framework is recent work done in the field of representing social connections within massive open online courses (MOOCs). Large online education providers, such as \textit{Coursera}, use thread messaging boards to facilitate collaboration between students. Within these forums, any student in a particular course can post a question or remark to start a thread of conversation. The data gathered from these educational messaging boards enables researchers to study the exchange of information between students \cite{Brinton2016}, the influence of students on others' participation in the course, \cite{Sinha2014a} and more general social dynamics of a class.

Sinha et al., for instance, model each participant within a particular \textit{Coursera} class as a node, and draw a directed edge from a thread or sub-thread initiator to any of the people who engaged in that discussion \cite{Sinha2014}. By doing so, the authors are able to represent the flow of topics and engagement within the social dynamics of a certain course. Sinha et al. were able to analyze measures of degree and betweenness centrality on this model of student engagement, in order to understand the influence of discussion initiators on the overall collaboration of students in the class. The parallels are made clear if we interpret discussion topics as self-contained units of knowledge that are part of a course. 

Similarly to our network construction, Sinha et al. use a gradient
coloring of nodes to indicate its relative betweenness
centrality. Zhu et al. also seek to model the engagement of
students in discussion forums on a week-by-week basis using
Exponential Graph Models (EGM) \cite{Zhu2016}. The use of EGMs allows
the authors to model a student's participation for a given week by
incorporating a student's performance for both the previous and
subsequent weeks. While inspired by EGMs, our model does not directly
include aggregate network statistics, in order to compute node and edge
properties. Instead, we model the probability of two courses
being taken sequentially given data in which the courses may have been
taken several academic terms apart. Finally, \textit{NetworkSeer} uses a similar modeling framework as in the previous papers but additionally models individual students' demographic information within the course discussion thread \cite{Wu2016}. Although we do not have direct access to students' demographic data, our model uses a student's final major to study the differences in course-taking behavior between students.

%Our course network takes inspiration from the general form of stochastic EGMs: $$P(Y=y) = \frac{1}{k(\theta)}exp(\theta^Tg(y)),$$ where Y is a random variable representing the network and y is a specific observed network.

The social dynamics of student participation in MOOCs and the spread
of knowledge through citation networks are the closest parallel to
modeling students' course-taking behavior in US universities. Courses can
be observed as distinct units of information that share overlap and
prerequisite knowledge with other courses. Academic publications and
forum discussions are similarly self-contained units of knowledge that
build upon and interact with other publications and discussion
threads, respectively. Given a dearth of direct research in course
sequence networks, our \textit{Via} toolkit builds upon the modeling
frameworks of citation graphs and MOOC social dynamic networks. In
contrast, course embeddings \cite{pard2018} represent courses and
enrollments as vectors, which are then clustered and otherwise
manipulated.

